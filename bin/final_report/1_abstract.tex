\chapter{Abstract}
\begin{tcolorbox}[title=Abstract,colbacktitle=white!80!black,coltitle=black,arc=0mm,boxrule=0.1mm]
A surface elastic wave is a wave travelling on the surface of a material without causing permanent changes. There are several factors affecting the propagation of these waves and the majority of them stem from physical material properties. A prime example of surface elastic waves are earthquakes, with their propagation being majorly dictated by ground composition. These waves can be broadly classified as $p$-waves or $s$-waves according to whether the wave is displaced in the direction of wave propagation, or in a direction orthogonal to the wave propagation, respectively. This report investigates the behaviour of $p$-waves through a simple material.


In doing so, the wave equation was derived from the equation of motion in order to obtain a value for wave speed in terms of density and Lamé coefficients. The 3D spherical solution to this equation was found in addition to a planar wave approximate solution. Finally, the governance of Snell's law in the propagation of $p$-waves through physical interfaces was explored.
We attempted to provide computational analysis of $p$-wave propagation through layered media to no avail.
\vspace{5mm}

\textbf{Key terms:} $p$-wave, seismic fault, wave propagation, wavefunction
\end{tcolorbox}