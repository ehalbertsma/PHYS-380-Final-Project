%\chapter{Abstract}
\begin{tcolorbox}[title=Abstract,colbacktitle=white!80!black,coltitle=black,arc=0mm,boxrule=0.1mm]
A few years ago, running magazines began publishing articles touting the benefits of beetroot juice consumption before endurance athletic events. Claims included increased aerobic oxygen uptake, and improved endurance.
In this literature review, we compare the response of elite and recreational distance running athletes to beetroot juice supplementation. 

Studies indicate that recreational athletes may benefit from both chronic consumption as well as acute consumption. A review by Dominguez states that acute supplementation of beetroot juice within 2-3 hours of exercise reduces oxygen consumption (at or below \votmax intensity), increases time to exhaustion, and may improve performance at \votmax intensity%\cite[13]{dominguez2017effects}
.

However, some newer studies suggest that these effects are insignificant in subjects performing at an elite level. Two studies are particularly relevant to elite athletes: one by Boorsma, University of Guelph, 2013; and the other by Balsalobre-Fern\'andez, Universidad Aut\'onoma de Madrid, 2018. The researchers conclude that beetroot juice offers no discernible performance benefit to distance running athletes competing at national level or above.
\vspace{5mm}

\textbf{Key terms: nitric oxide, nitrate, oxygen uptake, distance running, beetroot juice, endurance athletes} 
\end{tcolorbox}