\chapter{Analyzing performance in distance running}\label{ch:performancemetrics}
%------------------
\section{Glossary of performance metrics in distance running}
Table \ref{tab:performancemetrics} outlines some of the key methods by which performance and fitness of distance runners can be evaluated. Each metric has its own benefits and drawbacks, therefor an assortment of multiple metrics are typically used to give a more complete picture of the subject's abilities.
\begin{table}[h]
    \caption{Glossary of performance metrics in distance running}
    \centering
    \begin{tabular}{|p{40mm}rp{100mm}|}
    \hline
    \textbf{Metric} & \textbf{Abbrev.} & \textbf{Description} \\\hline
        Time trial or competition results &  & In the sport distance running, the ultimate measure of fitness is considered competition performance. In the typical format of a footrace, an athlete competes with the ultimate aim of minimizing their finishing time over a particular distance.\cite{daniels2013daniels}
\\
        Rating of perceived effort & RPE &     An athlete's individual, subjective rating of their own effort level, based on their own sensory feedback. The rating is typically assigned an integer between 0 and 10, where 0 is no effort and 10 is a maximal effort.\cite{daniels2013daniels}
\\
        Oxygen uptake & \vot & \vot{} is the rate at which an athlete's aerobic energy system consumes oxygen. An athlete's ability to process oxygen has an upper bound, known as \votmax{}. An increasing fraction of energy metabolized after this point is done through the anaerobic energy system. This metric is measured in \si{L/min}, or in \si{L/kg/min} when adjusted for the athlete's body mass.\cite{daniels2013daniels}
        
Related to this: \vcot{} is the rate at which an athlete expels carbon dioxide.
\\
        Respiratory exchange ratio & RER &  The respiratory exchange ratio describes the numerical ratio of an athlete's volume of inhaled oxygen to the volume of exhaled carbon dioxide (\vot divided by \vcot). This indicates how much of the oxygen is being consumed, and is a metric of fitness when compared against the effort level sustained by the athlete during the period of measurement.\cite{daniels2013daniels}
\\
        Heart rate & HR & Recorded in beats per minute (bpm), a human's heart rate is positively correlated to effort level. The heart's minimum rate is called the Resting Heart Rate (RHR), which occurs when the subject is resting and immobile for a period of time. The maximum heart rate (MHR) is the heart rate occurring at maximal effort, such as when sprinting up a hill. The benefit of heart rate is its ease of measurement, and several models exist to approximate the relationship between \vot{} and HR.\cite{daniels2013daniels}
\\
        Running economy &  & Running economy is similar to fuel economy in a vehicle. It is defined as an athlete's instantaneous oxygen consumption relative to their velocity. This gives a measurement of the body's efficiency in converting oxygen to physical work. It is typically measured in \SI{}{L\per min\per kg \per m \cdot s}, where \SI{}{m\per s} is the athlete's forward velocity.\cite{daniels2013daniels}\\
        Muscle oxygen saturation & \smo & Measured as a percentage, oxygen saturation is defined by the fraction of inhaled oxygen which is consumed by the muscles. A near-infrared spectroscopy sensor can be placed against the skin to obtain the measurement. \cite{balsalobre2018effects} \\
        Time to exhaustion & \tex & Time to exhaustion is measured through a test, by which a subject is asked to sustain a fixed physiological workload indefinitely. At some point, the subject will reach a point of exhaustion and fail to sustain the required output. The measured time interval up to the point of failure is called time to exhaustion. \cite{nicolo2019comparison} \\\hline
    \end{tabular}
    \label{tab:performancemetrics}
\end{table}

\iffalse
%------------------
\subsection{Competition or time trial results}\cn
In the sport distance running, the ultimate measure of fitness is considered competition performance. In the typical format of a footrace, an athlete competes with the ultimate aim of minimizing their finishing time over a particular distance.
%------------------
%\subsection{Lactate Threshold (LT)}
\subsection{Rating of Perceived Effort (RPE)}\cn
    An athlete's individual, subjective rating of their own effort level, based on their own sensory feedback. The rating is typically assigned an integer between 0 and 10, where 0 is no effort and 10 is a maximal effort.
%------------------
%\subsection{Ventilatory Threshold (VT)}
%------------------
\subsection{Oxygen uptake ($\mathbf{VO_\textsc{2}}$)}\cn
\vot{} is the rate at which an athlete's aerobic energy system consumes oxygen. An athlete's ability to process oxygen has an upper bound, known as \votmax{}. All energy metabolized after this point is done through the anaerobic energy system\cn. This metric is measured in \si{L/min}, or in \si{L/kg/min} when adjusted for the athlete's body mass. 

Similarly, \vcot{} is the rate at which an athlete expels carbon dioxide.
%------------------
\subsection{Respiratory Exchange Ratio (RER)}\cn
    The respiratory exchange ratio describes the numerical ratio of an athlete's volume of inhaled oxygen to the volume of exhaled carbon dioxide. This indicates how much of the oxygen is being consumed, and is a metric of fitness when compared against the effort level sustained by the athlete during the period of measurement.
%------------------
\subsection{Heart rate (HR)}\cn
    Recorded in beats per minute (bpm), a human's heart rate is positively correlated to effort level. The heart's minimum rate is called the Resting Heart Rate (RHR), which occurs when the subject is resting and immobile for a period of time. The maximum heart rate (MHR) is the heart rate occurring at maximal effort, such as when sprinting up a hill.
\subsection{Muscle oxygen saturation (\smo)}\todo{populate this}\cn
\subsection{Time to exhaustion (\tex)}\todo{populate this}\cn
\hyperref[https://link.springer.com/article/10.1007/s11332-019-00585-7]{study?}
%------------------
\subsection{Running economy}
Defined as an athlete's capacity to do work at a given oxygen consumption. This can be measured in watts per liter of consumed oxygen \si{W/\vot} \cite[9]{dominguez2017effects}, or \si{W/kg/\vot}. An athlete who is able to output the same amount of work at a lower fraction of their \votmax{} will be able to sustain that effort for a longer period of time.
%------------------

%\section{Enhancing performance}\todo[color=yellow]{Not sure whether this is relevant to include.}
%------------------
%\subsection{Known methods for increasing oxygen uptake}
%\begin{lstlisting}
%- overview of how training increases VO2
%- reference specific training adaptations such as mitochondrial density and increased capillary pathways
%\end{lstlisting}
%------------------
%\subsection{Doping}
%\begin{lstlisting}
%- EPO, a banned substance, increases oxygen uptake
%- effectiveness of EPO
%- beetroot juice has a similar but lesser effect on athletes
%\end{lstlisting}
\fi