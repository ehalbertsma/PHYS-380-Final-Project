\chapter{Conclusion}
%result
It is clear that consumption of beetroot juice has some positive effect on the performance of recreational distance running athletes. Consumption of beetroot juice both in a short time interval preceding a performance (2-5 hours), or longer-term supplementation in the week preceding, is shown to improve the athlete's running economy, time to exhaustion, and cardiorespiratory performance at the \votmax intensity level
\cite{dominguez2017effects}.

%key findings
However, through the work of Boorsma \cite{boorsma2013effect} and Balsalobre et al \cite{balsalobre2018effects}, it is clear that beetroot juice has a much weaker effect on elite runners. 
Elite runners seem not to benefit from beetroot juice suppelementation with respect to in \vot, \votmax, running economy, or timed performance over set distances. Some improvement has been observed to likely occur in RPE, muscle oxygen saturation, and time to exhaustion\cite{balsalobre2018effects}.

The results observed in elite male runners agree with prior findings in similar studies for elite flatwater kayakers\cite{muggeridge2013effects} and cyclists\cite{christensen2013influence}: beetroot juice supplementation has no clear  performance benefit for elite athletes.

It is recommended that future research investigate the cause for beetroot juice's diminished effectiveness in elite runners. Boorsma did find that a small fraction of athletes did show some statistically significant performance improvement, suggesting that there may be some variance among athletes. De Castro posits that elite athletes may have a higher production of $\mathrm{NO}$ through the endogenous pathway, leading to a reduced effectiveness of $\mathrm{NO}$ production via dietary nitrate\cite[13]{de2019effect}.